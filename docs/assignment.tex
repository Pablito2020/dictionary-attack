\documentclass[12pt, letterpaper]{article}
\usepackage{graphicx} % needed for images
\usepackage[export]{adjustbox} % needed for adjustable images
\usepackage{flafter} % needed for figures
\usepackage{hyperref} % needed for a clickable table of contents
\usepackage{polyglossia} % needed for catalan support
\usepackage{subfiles} % needed for a subfile structure
\usepackage[table]{xcolor}
\usepackage{fancyhdr} % needed for fancy header 
\usepackage{subfiles} % needed for a subfile structure
\usepackage{listings}
\usepackage{pdflscape}
\usepackage{tabularx}
\usepackage{array}
\usepackage{fontspec}
\usepackage{texlogos}
\usepackage{csquotes}
% Bibliography
\usepackage[sorting=nyt]{biblatex}
\usepackage{minted}
\DeclareBibliographyCategory{cited}
\AtEveryCitekey{\addtocategory{cited}{\thefield{entrykey}}}

\lstset{basicstyle=\footnotesize\ttfamily}
\lstset{framextopmargin=50pt,frame=bottomline}

\usepackage{color}
\usepackage{colortbl}
\usepackage{lineno}
\hypersetup{
    colorlinks=true,
    linkcolor=black,
    citecolor = blue,      
    urlcolor=blue,
}

\lstset{aboveskip=20pt,belowskip=20pt}

% \setsansfont{Calibri}
\setmonofont[Scale=0.7]{JetBrains Mono}

\setmainlanguage{catalan}
\graphicspath{ {images} }

\pagestyle{fancy}
\fancyhf{}
\fancyhead[LE,RO]{Seguretat d'Aplicacions i Comunicacions}
\fancyhead[RE,LO]{Atac de diccionari}
\fancyfoot[LE,RO]{\thepage}
\renewcommand{\headrulewidth}{1pt}
\renewcommand{\footrulewidth}{1pt}

% information
\title{%
    \begin{center}
	\includegraphics[width=4cm,height=3cm]{udl.png}
    \end{center}
    \line(1,0){250}\\[0.3cm]
    \textbf{Atac de diccionari} \\
    \line(1,0){250}
    \\[0.5cm]
	\large Seguretat d'Aplicacions i Comunicacions - Grau en Enginyeria Informàtica
}
\author{Pablo Fraile Alonso}
\date{\today}

% document
\begin{document}
    
% title
\maketitle
\thispagestyle{empty}
\newpage
\tableofcontents
\listoffigures
\newpage

\section{Procediment}
El procediment és simple, per tal de realitzar un atac de diccionari, per cada passphrase possible fem:
\begin{enumerate}
    \item Obtenir el salt i el missatge xifrat de l'arxiu \textit{exercici.bin}.
    \item Amb cada possible passphrase, calcular la clau que emprarem a AES a partir de l'output de la funció PBKDF2HMAC (amb algoritme SHA1 i el salt anterior) sobre la passphrase.
    \item Un cop calculada la clau, s'aplica el desxifratge del missatge amb l'algoritme AES (128 bits) amb mode ECB. Aquí hi han dos casos:
        \begin{itemize}
            \item En cas de que el missatge desxifrat \textbf{NO} es pugui decodificar en "utf-8", la donem com a invàlida, i per tant provem la següent passphrase.
            \item En cas de que la clau pugui decodificar el missatge desxifrat en utf-8, es mostra el missatge i la passphrase corresponent a la clau.
        \end{itemize}
\end{enumerate}

El procediment que es segueix amb els hash precomputats, és el següent:
\begin{enumerate}
    \item S'executa el script \textit{cache\_passwords.py} que genera un fitxer on hi han les passphrases i la seva respectiva clau\footnote{derivada amb la funció PBKDF2HMAC i el seed=salt} separades per coma.
    \item Un cop generat el fitxer, s'executa \textit{dictionary\_attack\_cached.py}, que desxifrarà el missatge amb cada clau del fitxer. Tornen a haver-hi els mateixos dos casos a l'hora d'analitzar el missatge desxifrat:
        \begin{itemize}
            \item En cas de que el missatge desxifrat \textbf{NO} es pugui decodificar en "utf-8", la donem com a invàlida, i per tant provem la següent clau.
            \item En cas de que la clau pugui decodificar el missatge desxifrat en utf-8, es mostra el missatge i la passphrase corresponent a la clau.
        \end{itemize}
\end{enumerate}

\section{Estructura del projecte}
L'estructura del projecte a \href{https://github.com/Pablito2020/dictionary-attack}{github} (la versió adjuntada al tar no conté els testos i altres fitxers de configuració) és la següent:
\begin{verbatim}
.
├── data
├── cache_passwords.py
├── dictionary_attack_cached.py
├── dictionary_attack.py
├── docs
├── src
│   ├── __init__.py
│   ├── password.py
│   ├── pkcs5.py
├── test
\end{verbatim}
On:
\begin{itemize}
    \item data: conté tots els arxius de dades, com ara el diccionari, l'arxiu binari de openssl amb el salt i el missatge, etc.
    \item src: conté classes útils per al projecte:
            \begin{itemize}
                \item \textit{password.py}: encapsula tot el relacionat amb les passphrases i generació de claus.
                \item \textit{pkcs5.py}: encapsula el parseig del fitxer i retorna una estructura de dades amb el missatge i el salt.
            \end{itemize}
    \item test: conté tests de les classes de src/
    \item docs: conté la documentació (en format latex) i l'enunciat de la pràctica.
    \item altres fitxers i directoris van relacionats amb la configuració de l'integració continua a github o bé configuració de programes que comproven el seguiment de bones pràctiques amb python (pylint, black, etc).
\end{itemize}

Tota l'informació necessària per a executar el projecte, es troba al fitxer \textbf{README.md}.

\section{Comparació de rendiment}
En cas de que executem l'atac \textit{"simple"} (és a dir, sense cap mena de caching amb les passphrases), veiem el següent:
\begin{verbatim}
[pablo@archcrypted ~/projects/dictionary] python dictionary_attack.py
PKCS5: 
[
	Salt is: 10f227276c4fc533,
	Message is: 464c08381439b4fd60377584e18e2da084b6d8ef84de2b9354f99f4df1a0eb9f
]
Message is: El password és Yukuo7
Password is: Yukuo7
Program finished in: 47.32314246299984
\end{verbatim}

En canvi, si executem l'atac amb les claus (derivades de la passphrase i el salt) ja calculades, veiem el següent resultat:
\begin{verbatim}
[pablo@archcrypted ~/projects/dictionary] python dictionary_attack_cached.py
PKCS5: 
[
	Salt is: 10f227276c4fc533,
	Message is: 464c08381439b4fd60377584e18e2da084b6d8ef84de2b9354f99f4df1a0eb9f
]
Message is: El password és Yukuo7
Password is: Yukuo7
Program finished in: 28.01076317000002
\end{verbatim}

Com podem veure, el fet de no haver de calcular la funció PBKDF2HMAC per cada passphrase ha decrementat en un 40\% el temps d'execució del programa (ha passat de 47 segons a 28).\\

Aquest mètode de fer cache de les claus, però, no es pot realitzar en mes d'un atac, ja que el salt sempre canviarà quan tornem a xifrar un altre missatge. Per tant, tot i que en aquest cas s'hagi guanyat un 40\% de rendiment, en casos \textit{"reals"} d'atac de diccionari\footnote{Com per exemple trobar la contrasenya d'una wifi amb hashcat} és gairebé impossible emprar aquesta tècnica.

\end{document}
